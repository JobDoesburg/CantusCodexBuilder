De cantus wordt geleid door het Praesidium. Het praesidium bestaat uit de Senior, de Procantor en de Punator. 

Er zijn twee belangrijke regels voor een cantus:
\begin{enumerate}
    \item Het Praesidium heeft altijd gelijk.
    \item Wanneer het Praesidium ongelijk heeft, geldt regel 1. 
\end{enumerate}

De Corona (de deelnemers aan de cantus) kan het Praesidium aanspreken.\\
Om dit te doen, kunnen de volgende zinnen worden gebruikt:

“\textbf{Senior, peto verbum?}”: Hiermee vraagt u aan de senior of u het woord mag.\\
“\textbf{Senior, peto tempus?}”: Hiermee vraagt u aan de senior of u naar het toilet mag gaan. U mag alleen gaan na het opzeggen van een limerick.  

De senior kan op twee manieren antwoorden, namelijk met “\textbf{habes}”, hiermee wordt toestemming gegeven, of met “\textbf{non habes}”, hiermee geeft hij geen toestemming. Als de senior toestemming heeft gegeven om te spreken, begint u uw speech met “\textbf{Geacht praesidium, geachte corona}”. 

Het praesidium heeft de macht om straffen op te leggen. De straffen moeten zonder protest worden aanvaard door de corona. Zodra er bij een straf gedronken dient te worden, dient u voorafgaand aan het nuttigen van de drank te proosten. De juiste manier om te proosten is als volgt:“\textbf{Prosit senior, prosit praesidium, prosit corona}”, gevolgd door de naam van de straf. Vervolgens ondergaat u de straf met sportief gedrag. 

Er wordt tijdens het zingen niet gedronken, tenzij het nuttigen van de drank wordt opgelegd door het praesidium.  


\subsubsection*{Enkele veelgebruikte termen:}
\textbf{Ad fundum} = “Drink het glas tot op de bodum”\\
\textbf{Ad libitum} = “Drink zoveel u wilt"\\
\textbf{Silentium} = “Stilte!”\\
\textbf{Sedet} = “Ga zitten!”\\\\
