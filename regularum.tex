\subsubsection*{Praesidium}
\footnotesize

\begin{enumerate}
    \item De cantus wordt geleid door het Praesidium.
    \item Binnen het Praesidium worden de volgende functies beoefend: Senior, Procantor en Punator.
    \item De functies Senior, Procantor en Punator worden toegewezen aan ten minste één en ten hoogste drie natuurlijk personen, al wie samen het Praesidium vormen.
    \item Het Praesidium staat boven de spreekwoordelijke wet: zij hebben altijd gelijk en bevinden zich nimmer in het ongewisse. 
    \item Mocht een lid uit de Corona zichzelf willen toewijzen als Senior, kan dit door het uitspreken van magische woorden 'Senior, peto Senior' uit te spreken. Afhankelijk van de stroefheid van de timing van dit edict zal er een overwegende 'Habes' of 'Non habes' volgen vanuit de Senior, waarmee hij respectievelijk wel of geen toestemming verleent. Een gelijkend proces geldt voor de functie van Procantor evenals Punator. 
    \item Wanneer de Senior zijn zin eindigt op 'Corona' in een toesprekenede vorm, dient de corona deze zelfde zin te exclameren, waarbij 'Corona' vervangen wordt door 'Senior'. Uitgezonderd hierop zijn drankadviezen.
\end{enumerate}

\subsubsection*{Corona}
\begin{enumerate}
    %\setcounter{enumi}{5}
    \item De cantus wordt belijdt door de Corona.
    \item Ieder aanwezig natuurlijk persoon die geen lid is van het Praesidium, is onderdeel van de Corona. 
    \item Alle leden van de Corona zijn elkaars gelijken, geen individu staat boven een ander, thans beschouwt de Corona zichzelf, alsmede wordt de Corona door het Praesidum beschouwd als een collectief orgaan. Ieder lid van de Corona draagt een verantwoordelijk jegens dit groepsgevoel.
    \item De Corona laat zich leiden door, en is onderworpen aan de wil van het Praesidium.
\end{enumerate}

\subsubsection*{Hi\"erarchie en Orde}
\begin{enumerate}
    \item De Senior staat in voor het tucht- en stijlvolle verloop van de cantus. Hij regelt het verloop ervan, bepaalt de samenzangen en rondgezangen. Daarnaast mag hij Silentium bevelen, Colloquium, Verbum en Tempus verlenen en straffen opleggen, alsmede het uitvaardigen van drink- en drankadviezen.
    \item Ieder lid van het Praesidium, krachtens goedkeuring van de Senior, kan handelen in naam van de Senior. Meer specifiek zal de Procantor zich hoofdzakelijk bezighouden met het samen- en rondgezang en de introductie van zulks. De Punator heeft daarentegen bijzonder oog voor een tuchtelijk verloop van de cantus en kan straffen opleggen en houdt zich bezig met de praktische uitvoering van opgelegde straffen.
    \item Een drankadvies kan gegeven worden in verscheidene vormen. Bij een 'Ad Fundum' ledigt de Corona haar glazen volledig, bij een 'Ad libitum' drinkt men al naar gelange.
    \item Het is niet toegestaan te drinken buiten de adviezen of straffen van het Praesidium. 
    \item Alvorens gedronken wordt, proost ieder lid met de woorden: "Prosit Senior, Prosit Procantor, Prosit Punator, Prosit Corona" gevolgd door een herhaling van het drank advies ("Ad Fundum" cq. "Ad Libitum") waarbij telkens de daarbij benoemde persoon in de ogen wordt aangekeken en een toostend gebaar wordt gemaakt.
    \item Elk edict van het Praesidium dient ogenblikkelijk gevolgd te worden.
\end{enumerate}

\subsubsection*{Codex en Gezangen}
\begin{enumerate}
    \item Aan het begin van de cantus wordt aan ieder lid van de Corona een Codex overhandigd, welke liedteksten van verschillende gezangen, alsmede het regularum bevat.
    \item Omwille van een deugdelijk verloop van de cantus is het ten strengste verboden deze codex te verliezen.
    \item De liedteksten in de Codex dienen strikt gevolgd te worden. Tevens wordt uitgegaan van een redelijk intelligtentieniveau van de Corona, zodat zij de aanwijzingen in de liedteksten correct interpreteren. Een verkeerde interpretatie kan worden bestraft door het Praesidium.
    \item Aan het begin van elk lied dient het voor iedereen duidelijk te worden gemaakt op welke pagina de tekst staat. Een voorbeeld hiervoor is "Pagina 10, één nul, nul één, pagina 10". De Corona draagt gezamenlijke verantwoordelijkheid voor deze aanduiding, tenzij het Praesidium deze verantwoordelijkheid heeft gegeven aan een expliciet individu. 
    \item Tijdens de gezangen is het nuttigen van drank niet toegestaan, tenzij zulks wordt voorgeschreven door de liedtekst.
    \item Buiten de zangen is het strikt verboden om de codex open te hebben liggen. 
\end{enumerate}

\subsubsection*{Verbum en Tempus}
\begin{enumerate}
    \item Een lid van de Corona kan een 'Verbum' aanvragen als hij een speech wil houden. Een 'Tempus' kan aangevraagd worden als men de Corona voor enkele ogenblikken wil verlaten, bijvoorbeeld voor een sanitair doeleinde.
    \item Voor het aanvragen van een Verbum dient een lid van de Corona te gaan staan en met een geestig handgebaar duidelijk te maken aan het Praesidium wie het woord wilt. Hierbij dienen de woorden "Senior, peto Verbum!" uitgesproken te worden. De Senior richt zich vervolgens tot dit individu met respectievelijk de woorden ‘Habes!’ of ‘Non habes!’, voor toewijzing of afwijzing.
    \item Bij een 'Habes!' zoals bedoeld in vorig artikel, bedankt het lid van de Corona het Praesidium voor het woord op adequate wijze alvorens een coherent verhaal te houden. Bij een 'Non habes!' neemt het individu weer zittend plaats en zwijgt zij. 
    \item Het aanvragen van een Tempus geschiedt op een gelijke wijze als het aanvragen van een Verbum, zij het met de term 'Tempus'. Echter dient de aanvraag van het Tempus vervolgd te worden met een limerick (rijmschema AABBA), waarna de Senior tot een definitief ‘Habes!’ of ‘Non habes!’ besluit. 
    \item Ten minste tweemaal gedurende de Cantus verleent de Senior een zogeheten Tempus Commune. Tijdens dit Tempus Commune heeft de gehele Corona gelegenheid te toiletteren. 
\end{enumerate}

\subsubsection*{Enkele straffen}
Het Praesidium, en effectief de Punator, put uit een grote originaliteit bij het opleggen van straffen welke bij opleggging beschreven zullen worden aan de gestrafte(n) uit de Corona. Hieronder volgen echter enkele straffen waarvan wordt aangenomen dat de Corona bekend is met de inhoud.

\begin{itemize}
    \item[\textbf{Ad Fundum}] De gestrafte dient binnen ongeveeer zeven seconden het glas te ledigen.
    
    \item[\textbf{Ciske de Ad}] Als een normaal Ad Fundum, doch nu dient de gestrafte het \textit{Ciske de Ad-lied} te caneren en mede-corona-lid uit te nodigen. Zij zullen samen, op de omschreven wijze in dit lied, een Ad Fundum nemen.
    
    \item[\textbf{Advies}] De gestrafte dient aan een door hemzelf aan te wijzen aanwezige te vragen op welke manier het adje uitgevoerd dient te worden. De gestrafte dient dit advies op te volgen.

    \item[\textbf{Adjehalvemeter}] De gestrafte krijgt een Ad Fundum toegereikt door een door het Praesidium uit te kiezen individu vanaf anderhalvemeter afstand.
    
    \item[\textbf{SchildpAd}] Deze straf vereist twee gestraften. De gestaften dienen op elkaar te gaan liggen, beiden met hun navel naar de vloer wijzend, waarna ze elkaar het bier in elkanders mond laten vloeien tot de bodem bereikt is.
    
    \item[\textbf{‘t Ad-vindertje}] Voor lieden die volledig de weg kwijt zijn. De gestrafte wordt geblinddoekt en neemt plaats op handen en voeten. Het glas wordt op de grond geplaatst, op een willekeurige locatie in het xy-vlak tussen de Corona en het Praesidium. De gestrafte dient zijn glas op te zoeken, waarbij het de Corona vrij staat advies te geven middels de woorden ‘warm’ en ‘koud’, evenals de comparativus van deze woorden. Vanaf het vinden van het glas volgt het normale Ad Fundum.
    
    \item[\textbf{Adje stoere boy}] De gestrafte krijgt de mogelijkheid om zich van zijn huidige kledij te ontdoen. Na de vrije invulling van deze mogelijkheid, dient de gestrafte een aantal glazen te nuttigen gelijkend aan het aantal kledingstukken wat nog bevestigd zijn aan diens lichaam. De hoeveelheid glazen per kledingstuk kan aangepast worden door het Praesidium indien dit wenselijk is in de situatie. 
    
    \item[\textbf{Adje stoerste boy}] Deze straft vereist twee gestraften. Vanaf het moment van oplegging is het beide gestraften ten strengste verboden verbaal danwel met gebaren te communiceren. Vervolgens dienen beide gestraften elkaar de rug te keren en ontdoen zij zich naar gelange van hun kledij. Na de vrije invulling van deze mogelijkheid, drinkt iedere gestrafte 1 glas per kledingstuk dat de ander uitgetrokken heeft. Indien beiden zich volledig van al hun kledij hebben ontdaan, echter, drinken beiden niets. De hoeveelheid glazen per kledingstuk kan aangepast worden door het Praesidium indien dit wenselijk is in de situatie. 

    \item[\textbf{SquAd}] De gestrafte dient tien squats te doen en gedurende deze gezonde lichamelijke oefening zijn adje te nuttigen in een evenredig tempo. 
    
    \item[\textbf{Ad Noot Mies}] De gestrafte mag vanaf nu louter woorden edicteren met een maximum van drie lettergrepen.
    
    \item[\textbf{CapitulAdtie}] De gestrafte mag vanaf nu louter drinken terwijl zijn rechterhand gestrekt in de lucht is.
    
    \item[\textbf{VindicAd}] De gestrafte wordt gedurende de rest van de avond gefeut door de rest van de aanwezigen.
    
    \item[\textbf{Body shAdt}] Er wordt gedronken uit de navel van een daartoe aangewezen persoon. De hoeveelheid die gedronken dient te worden, wordt bepaald door het Praesidium afhankelijk van de volumecapaciteit van de betreffende navel. 

    \item[\textbf{Guantanamo Ad}] Een gestrafte wordt vastgebonden op zijn stoel waarna daartoe door het Praesidium aangewezen leden van de Corona op ludieke wijze drank voeren aan de gestrafte. Het is de gestrafte niet toegestaan zijn handen te gebruiken.

    \item[\textbf{Two Girls One Ad}] Deze straft vereist twee gestraften. Een gestrafte vult zijn mond met bier alvorens deze over te brengen in de mond van de andere gestrafte die het vervolgens doorslikt. Dit wordt herhaald tot het glas geledigd is. Vervolgens gebeurt vice versa hetzelfde voor de andere gestrafte. 

\end{itemize}
