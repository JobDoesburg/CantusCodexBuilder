The cantus will be led by the Praesidium. The Praesidium consists of the Senior, the Procantor and the Punator.

There are two main rules for a cantus:
\begin{enumerate}
    \item The Praesidium is always right.
    \item When the Praesidium is wrong, rule 1 applies.
\end{enumerate}

The Corona (the participants in the cantus) can address the Praesidium.\\
To do this, the following statements can be used:

“\textbf{Senior, peto verbum?}”: With this you ask the Senior if you can speak.\\
“\textbf{Senior, peto tempus?}”: With this you ask the Senior if you can go to the toilet. You may only go after reciting a limerick.  

The Senior can answer in two ways, namely with “\textbf{habes}”, which gives permission, or with “\textbf{non habes}”, which does not give permission. If the Senior has given permission to speak, start your speech with “\textbf{I thank you for the word Senior, Dear Senior, Dear Praesidium, Dear Corona}”.

The Praesidium has the power to impose penalties. The penalties must be accepted without protest by the Corona. As soon as drinking is required during a punishment, you should toast before consuming the drink. The correct way to toast is as follows:“\textbf{Prosit Senior, Prosit Praesidium, Prosit Corona}”, followed by the name of the punishment. You then undergo the punishment with sportsmanlike behavior.

There is no drinking while singing, unless the consumption of the drink is ordered by the Praesidium.

\subsubsection*{Some commonly used terms:}
\begin{enumerate}
\item[\textbf{Ad fundum}] “Drink the glass to the bottom”
\item[\textbf{Ad libitum}] “Drink as much as you want”
\item[\textbf{Silentium}] “Silence!”
\item[\textbf{Sedet}] “Sit down!”
\end{enumerate}
