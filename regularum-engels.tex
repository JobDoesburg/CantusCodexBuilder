\subsubsection*{Praesidium}
\begin{enumerate}
    \item The cantus will be lead by the Praesidium.
    \item Within the Praesidium there are the following functions: Senior, Procantor en Punator.
    \item The functions of Senior, Procantor and Punator will be assigned to at least one person and a maximum of three natural people, which will all form the Praesidium. 
    \item The Praesidium is above the proverbial law: they are always right and are never wrong\label{always-right}.
    \item Whenever the Praesidium is wrong, rule \ref{always-right} applies.
    \item Whenever the Senior ends his sentence with ``\textit{Corona}'' in speech, the Corona should declare this same sentence, replacing ``\textit{Corona}'' with ``\textit{Senior}''. Exclamations of drinking are excluded from this.
    \item The Praesidium can appoint so-called `\textit{Wardens}'. They serve to support the leadership of the Praesidium and supervise the Corona and can also help with practical support for the smooth running of the cantus. Wardens are therefore authorized to act on behalf of the Praesidium, but an edict of a Warden can be nullified by the Praesidium.
     \item The Corona must treat Wardens as an integral part of the Praesidium.
\end{enumerate}


\subsubsection*{Corona}
\begin{enumerate}
    \setcounter{enumi}{6}
    \item The cantus will be avowed by the Corona.
    \item Every present natural person that is not a member of the Praesidium, is a member of the Corona.
    \item All members of the Corona are equal, no individual is above another, as the Corona considers themselves. However, by the Praesidium, the Corona is seen as a collective organ. Every member of the Corona bears a responsibility towards this group feeling. 
    \item The Corona will be led by and subjected to the wishes of the Praesidium. 
\end{enumerate}


\subsubsection*{Hierarchy and Order}
\begin{enumerate}
    \setcounter{enumi}{10}
    \item The Senior is responsible for the disciplinary and smooth course of the cantus. He regulates its transition and determines the congregational chants. In addition, he may order \textit{Silentium}, grant \textit{Colloquium}, \textit{Verbum} and \textit{Tempus} and impose punishments, as well as issuing drinking and hand out drinking advices.
    \item Any member of the Praesidium, by virtue of the Senior's approval, may act on the Senior's behalf. More specifically, the Procantor is mainly concerned with singing together and the introduction of every song. The Punator, on the other hand, has a special eye for the disciplinary conduct of the cantus and can impose punishments. He is also concerned with the practical implementation of the imposed punishments.
    \item Drinking advice can be given in various forms. With an 'Ad Fundum' the Corona empties her glasses completely, and with an 'Ad libitum' people drink according to their preference.
    \item It is not allowed to drink outside the advices or punishments of the Praesidium.
    \item The Praesidium can declare any member of the Corona to be `alcohol impotent'. Those who are `alchol impotent' are not allowed to drink under any circumstances.
    \item Before drinking, each member will toast with the following words: ``\textit{Prosit Senior, Prosit Procantor, Prosit Punator, Prosit Corona}'' followed by a repetition of the drink advice (``\textit{Ad Fundum}'' or ``\textit{Ad Libitum}''), whereby the corresponding named person is looked in the eye and a toasting gesture is made.
    \item Any edict of the Praesidium should be followed immediately.
\end{enumerate}


\subsubsection*{Codex and Chants}
\begin{enumerate}
    \setcounter{enumi}{16}
    \item At the beginning of the cantus, every member of the Corona is handed over a Codex, which contains lyrics of various chants, as well as the Regularum.
    \item For the sake of a proper course of the cantus, it is strictly prohibited to lose track of this codex.
    \item The lyrics in the Codex must be strictly followed. A reasonable level of intelligence of the Corona is assumed, so that they shall correctly interpret the directions in the song texts. Misinterpretation may be penalized by the Praesidium.
    \item At the beginning of each song, it should be made clear to everyone on which page the lyrics are located. An example for this is ``\textit{Page ten, one zero, zero one, page ten}''. The Corona bears joint responsibility for this designation, unless the Praesidium has given this responsibility to an explicit individual.
    \item Drinking is not allowed during the chants, unless this is prescribed by the lyrics or determined by the Praesidium.
    \item When there is no singing, it is strictly forbidden to have the codex open.
\end{enumerate}


\subsubsection*{Verbum and Tempus}
\begin{enumerate}
    \setcounter{enumi}{22}
    \item Each member of the Corona can request a ``\textit{Verbum}'' if they want to give a speech. A ``\textit{Tempus}'' can be requested if one wishes to leave the Corona for a moment, for example for sanitary purposes.
    \item To request a ``\textit{Verbum}'', a member of the Corona must stand and indicates to the Praesidium who wants to speak, while doing a witty hand gesture. The words ``\textit{Senior, peto Verbum!}'' must be spoken. The Senior then addresses this individual with the words ``\textit{Habes!}'' or ``\textit{Non habes!}'' respectively, for approval or rejection.
    \item At a ``\textit{Habes!}'', as referred to in the previous article, the member of the Corona must thanks the Praesidium appropriately for being allowed to talk before giving a coherent story. At a ``\textit{Non Habes!}'' the individual sits down again and shall remain silent.
    \item Requesting a ``\textit{Tempus}'' is done in a similar fashion as for requesting a ``\textit{Verbum}'', albeit with the term `\textit{Tempus}'. However, the request of the ``\textit{Tempus}'' must be continued with a limerick (rhyme scheme AABBA), after which the Senior decides on a definitive ``\textit{Habes!}'' or ``\textit{Non habes!}''.
    \item At least twice during the cantus, the Senior shall grant a so-called ``\textit{Tempus Commune}''. During this ``\textit{Tempus Commune}'', the entire Corona has the opportunity to visit the toilet.
\end{enumerate}


\subsubsection*{Examplary punishments}
%Het Praesidium, en effectief de Punator, put uit een grote originaliteit bij het opleggen van straffen welke bij opleggging beschreven zullen worden aan de gestrafte(n) uit de Corona. Hieronder volgen echter enkele straffen waarvan wordt aangenomen dat de Corona bekend is met de inhoud.

The Praesidium, more specifically the Punator, draws from great originality in the imposition of their penalties. These punishments will be imposed upon to the punished of the Corona.

\begin{itemize}
    \item[\textbf{Ad Fundum}] %De gestrafte dient binnen ongeveeer zeven seconden het glas te ledigen.
    The punished needs to clear their glass within seven seconds.
    
    \item[\textbf{Ciske de Ad}] %Als een normaal Ad Fundum, doch nu dient de gestrafte het \textit{Ciske de Ad-lied} te caneren en mede-corona-lid uit te nodigen. Zij zullen samen, op de omschreven wijze in dit lied, een Ad Fundum nemen.
    See \textbf{Ad Fundum}, with the addition that the punished needs to sing the \textbf{Ciske de Ad-lied} and needs to invite a follow Corona members to join them. They both shall then, following the practises described in the song, take their Ad Fundum. 
    
    \item[\textbf{Advies}] %De gestrafte dient aan een door hemzelf aan te wijzen aanwezige te vragen op welke manier het adje uitgevoerd dient te worden. De gestrafte dient dit advies op te volgen.
    The punished shall ask someone else - appointed by the punished - what punishment they need to perform, and shall then perform this punishment.

    \item[\textbf{Adjehalvemeter}] %De gestrafte krijgt een Ad Fundum toegereikt door een door het Praesidium uit te kiezen individu vanaf anderhalvemeter afstand.
    The punished shall receive an Ad Fundum at the distance of one and a half meter from another individual appointed by the Praesidium. 
    
    \item[\textbf{SchildpAd}] 
    %Deze straf vereist twee gestraften. De gestaften dienen op elkaar te gaan liggen, beiden met hun navel naar de vloer wijzend, waarna ze elkaar het bier in elkanders mond laten vloeien tot de bodem bereikt is.
    This punishment requires two members that need to be punished. The punished need to lie on top of each other, both with their belly button facing the floor. After this, they shall let the beer flow in each others mouth until the glasses are empty. 
    
    \item[\textbf{‘t Ad-vindertje}] %Voor lieden die volledig de weg kwijt zijn. De gestrafte wordt geblinddoekt en neemt plaats op handen en voeten. Het glas wordt op de grond geplaatst, op een willekeurige locatie in het xy-vlak tussen de Corona en het Praesidium. De gestrafte dient zijn glas op te zoeken, waarbij het de Corona vrij staat advies te geven middels de woorden ‘warm’ en ‘koud’, evenals de comparativus van deze woorden. Vanaf het vinden van het glas volgt het normale Ad Fundum.
    For those who are pretty far gone. The punished shall be blindfolded and shall place both hands and feet on the floor. The glass will be positioned on the floor, on a random location in the xyz-surface between the Corona and the Praesidium. The punished needs to look for their glass, in which the Corona is free to extend advice with the words ''hot'' and ''cold'', or with similar words. When the punished has found the glass, they have to perform a normal Ad Fundum.
    
    
    \item[\textbf{Adje stoere boy}] %De gestrafte krijgt de mogelijkheid om zich van zijn huidige kledij te ontdoen. Na de vrije invulling van deze mogelijkheid, dient de gestrafte een aantal glazen te nuttigen gelijkend aan het aantal kledingstukken wat nog bevestigd zijn aan diens lichaam. De hoeveelheid glazen per kledingstuk kan aangepast worden door het Praesidium indien dit wenselijk is in de situatie. 
    The punished is given the opportunity to undress themselves of their current clothing. After the free interpretation of this request, the punished is given a number of glasses that they have to drink. This number is directly connected to the number of clothing still attached to the punished body. The number of glasses per article of clothing can be changed as the Praesidium sees fit for the opportunity.
    
    \item[\textbf{Adje stoerste boy}] %Deze straft vereist twee gestraften. Vanaf het moment van oplegging is het beide gestraften ten strengste verboden verbaal danwel met gebaren te communiceren. Vervolgens dienen beide gestraften elkaar de rug te keren en ontdoen zij zich naar gelange van hun kledij. Na de vrije invulling van deze mogelijkheid, drinkt iedere gestrafte 1 glas per kledingstuk dat de ander uitgetrokken heeft. Indien beiden zich volledig van al hun kledij hebben ontdaan, echter, drinken beiden niets. De hoeveelheid glazen per kledingstuk kan aangepast worden door het Praesidium indien dit wenselijk is in de situatie. 
    This punishment requires two people that need to be punished. When this punishment starts, both participants are not allowed to communicate with each other. Both then turn their backs towards each other and shall remove themselves of their clothing as they see fit. After the free interpretation of this opportunity, both punished shall drink one glass per article of clothing that the other took off. In case both have rid themselves of all their clothing, both will not have to drink anything. The number of glasses per article of clothing can be changed as the Praesidium sees fit given the situation. 

    \item[\textbf{SquAd}] %De gestrafte dient tien squats te doen en gedurende deze gezonde lichamelijke oefening zijn adje te nuttigen in een evenredig tempo. 
    The punished will need to perform ten squats and while performing this healthy exercise needs to finish their glass in a similar tempo. 
    
    \item[\textbf{Ad Noot Mies}] %De gestrafte mag vanaf nu louter woorden edicteren met een maximum van drie lettergrepen.
    The punished is only allowed to utter words up to three syllables.
    
    \item[\textbf{CapitulAdtie}] %De gestrafte mag vanaf nu louter drinken terwijl zijn rechterhand gestrekt in de lucht is.
    The punished is only allowed to drink while their right hand reaches out to the sky.
    
    \item[\textbf{VindicAd}] %De gestrafte wordt gedurende de rest van de avond gefeut door de rest van de aanwezigen.
    The punished shall be hazed by the rest of the attendees for the remainding duration of the cantus.
    
    \item[\textbf{Body shAdt}] %Er wordt gedronken uit de navel van een daartoe aangewezen persoon. De hoeveelheid die gedronken dient te worden, wordt bepaald door het Praesidium afhankelijk van de volumecapaciteit van de betreffende navel. 
    The punished shall drink from the belly button of a member appointed by the Praesidium. The amount that needs to be consumed will be decided by the Praesidium, depending on the volume capacity of said belly button.

    \item[\textbf{Guantanamo Ad}] %Een gestrafte wordt vastgebonden op zijn stoel waarna daartoe door het Praesidium aangewezen leden van de Corona op ludieke wijze drank voeren aan de gestrafte. Het is de gestrafte niet toegestaan zijn handen te gebruiken.
    The punished will be bound to a chair after which the Praesidium appoints members of the Corona to make sure the punished drink the drink in a way they see fit.

    \item[\textbf{Two Girls One Ad}]% Deze straft vereist twee gestraften. Een gestrafte vult zijn mond met bier alvorens deze over te brengen in de mond van de andere gestrafte die het vervolgens doorslikt. Dit wordt herhaald tot het glas geledigd is. Vervolgens gebeurt vice versa hetzelfde voor de andere gestrafte. 
    This punishment requires two members of the Corona to be punished. One punished fill their mouth with their drink to subsequently transfer this liquid into the mouth of the other punished, which then has to consume the liquid. This will also be performed the other way around, ensures both will empty a glass.

\end{itemize}