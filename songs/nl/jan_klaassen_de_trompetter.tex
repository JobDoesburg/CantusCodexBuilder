\section{Jan Klaassen De Trompetter}
\textbf{Refrein:}\\
Jan Klaassen was trompetter in het leger van de prins\\
Hij marcheerde van Den Helder tot Den Briel\\
En hij had geen geld en hij was geen held\\
En hij hield niet van het krijgsgeweld\\
Maar trompetter was hij wel in hart en ziel

Het leger sloeg z'n tenten op voor Alkmaar in het veld\\
En zolang geen vijand zich liet zien was iedereen een held\\
De kroeg werd als strategisch punt door 't hoofdkwartier bezet\\
De officieren brulden: Jan, kom speel op je trompet\\
Ze werden wakker in de goot, in de morgen kil en koud\\
Maar Jan Klaassen sliep in de armen van de dochter van de schout

\textbf{Refrein}

De prins sprak op inspectie tot de majoor van de compagnie\\
Ik zie hier alle stukken wel van de artillerie\\
Ja, zelfs dat kleine in uw kraag en dat blonde in uw bed\\
Maar waar zit dat stuk ongeluk van een Jan met z'n trompet

En niemand die Jan Klaassen zag die bij de stadspoort zat\\
En honderd liedjes speelde voor de kinderen van de stad

\textbf{Refrein}

Jan Klaassen zei: Vaarwel mijn lief, ik zie je volgend jaar\\
Wanneer de lente terugkomt dan zijn wij weer bij elkaar\\
De winter ging, de zomer kwam, de oorlog was voorbij\\
Maar 't leger is nooit teruggekeerd van de Mokerhei\\
Geen mens die van Jan Klaassen ooit iets teruggevonden heeft\\
Maar alle kinderen kennen hem, hij is niet dood, hij leeft

\textbf{Refrein}

En hij had geen geld en hij was geen held\\
En hij hield niet van het krijgsgeweld\\
Maar trompetter was hij wel in hart en ziel
